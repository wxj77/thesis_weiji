
\chapter{Binomial likelihood discussion}
If we measure a sequence of events, that we know is drawn from a Bernoulli distribution. 
How would we estimate the probability of the Bernoulli distribution $p$?
Assuming measuring this Bernoulli experiment $N$ times, and the result is ${X_1=x_1, ..., X_N=x_n}$. 
Among these events, value 1 happen $m$ times. 
Then the probability for this result to happen if $p$ is the Bernoulli probability is

\begin{equation}
P(X_1=x_1, ... X_N=x_n|p)=p^m(1-p)^{N-m}
\label{a1}
\end{equation}.

According to Bayes Rules,
\begin{equation}
P(A|B)=\frac{P(B|A)P(A)}{P(B)}
\label{a2}
\end{equation},

\begin{equation}
P(p|X_1=x_1, ... X_N=x_n)=\frac{P(X_1=x_1, ... X_N=x_n|p)P(p)}{P(X_1=x_1, ... X_N=x_n)}
\label{a3}
\end{equation}

The probability of $P(X_1=x_1, ... X_N=x_n)$ is a constant unrelated to $p$.
The probability of $P(p)$ would also be a constant if we know no further information,

\begin{equation}
f(p)=1,\quad p \in[0,1]
\label{a4}
\end{equation}.

Combining \ref{a1}, \ref{a3}, \ref{a4}, we get

\begin{equation}
P(p|X_1=x_1, ... X_N=x_n) \propto  P(X_1=x_1, ... X_N=x_n|p)
\label{a5}
\end{equation}.

Considering the normalization of probability, we get

\begin{equation}
P(p|X_1=x_1, ... X_N=x_n) =\frac{p^m(1-p)^{N-m}}{\int_0^1 dp\quad p^m(1-p)^{N-m}}
\label{a6}
\end{equation}.

Maximum of this probability, we get
\begin{equation}
\max_p P(X_1=x_1, ... X_N=x_n|p) = 
\max_p p^m(1-p)^{N-m} =\frac{m}{N}
\label{a7}
\end{equation}
Expected value of p give ${X_1=x_1, ..., X_N=x_n}$,
\begin{equation}
E(p|X_1=x_1, ... X_N=x_n) =\frac{\int_0^1 dp\quad p\quad p^m(1-p)^{N-m}}{\int_0^1 dp\quad p^m(1-p)^{N-m}}
=\frac{1+m}{2+N}
\label{a8}
\end{equation}.

\begin{equation}
\lim_{m,N \rightarrow \infty} E(p|X_1=x_1, ... X_N=x_n) 
\approx \frac{m}{N}
\label{a9}
\end{equation}.

Variance of p give ${X_1=x_1, ..., X_N=x_n}$,
\begin{align}
Var(p|X_1=x_1, ... X_N=x_n)& =E(p^2|X_1=x_1, ... X_N=x_n)-E(p|X_1=x_1, ... X_N=x_n)^2 \nonumber\\
& =\frac{\int_0^1 dp\quad p^2\quad p^m(1-p)^{N-m}}{\int_0^1 dp\quad p^m(1-p)^{N-m}} \nonumber
-(\frac{\int_0^1 dp\quad p\quad p^m(1-p)^{N-m}}{\int_0^1 dp\quad p^m(1-p)^{N-m}})^2 \nonumber\\
&=\frac{(1+m)(1+N-m)}{(2+N)^2(3+N)}
\label{a10}
\end{align}.

\begin{equation}
\lim_{m,N \rightarrow \infty} Var(p|X_1=x_1, ... X_N=x_n) 
\approx \frac{m(N-m)}{N^3} = \frac{\frac{m}{N}(1-\frac{m}{N}
)}{N}
\label{a11}
\end{equation}.

We realize that the probability expected value is not exactly equal to $m/N$, which maximized the likelihood function and also is the mean value of the results. Variance is not exactly $m/N(1-m/N)/N$. This is because the origin guess for we have $p$ is uniformly distributed between 0 and 1. We can not eliminate the small probabilities for value of $p$ deviated a lot from expected value. 