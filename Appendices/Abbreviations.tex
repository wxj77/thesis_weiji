\chapter{Abbreviations}
\label{chapter:abbrev}
This chapter summarizes the abbreviations that occur in this thesis.

\#: the counts of

$\sim$: approximately

ADC: Analog-to-Digital Converter

avg.: average

BBN: Big Bang Nucleosynthesis

CCD: Charge-couple device

CDF:  Cumulative Distribution Function

CMB: Cosmic Microwave Background

config.: configuration

cont.: continued

CSDA range: Continuous Slowing Down Approximation range

CWW: Coincidence Window Width

CV: Coefficient of Variation

DAQ: Data AcQuisition

DM: Dark Matter

dur.: duration

EL: ElectroLuminescence

ELD: ElectroLuminescence Detector

ER: Electron Recoil (event)

LUX: Large Underground Xenon experiment

LZ: LUX-ZEPLIN experiment

max.: maximum

min.: minimum

MFC: Mass Flow Controller

NR: Nuclear Recoil (event)

PDE: Photon detection efficiency (also called light collection efficiency)

PDF: Probability Distribution Function

PMF: Probability Mass Function

PEEK: PolyEther Ether Ketone

PHD(phd): counts of PHotoelectrons Detected

PHE(phe): \sphe\ pulse area or counts of (single) PHotoElectrons. In other literatures, it is sometime called PE(pe).

PMT: Photomultiplier Tube

PPB(ppb): parts per billion atoms/molecules

PTFE: PolyteTraFluoroEthylene

R\&D: Research and Development

refl.: reflectivity

RQ: Reduce Quantity of a pulse

\sone : primary Scintillation light

\stwo : secondary Scintillation light

S, SF: Survival Function

SS: Stainless Steel

QE: Quantum Efficiency (of a PMT)

QF: Quiet Fraction

SLAC: SLAC national accelerator laboratory

TBA: Top-Bottom Asymmetry

TPC: Time Projection Chamber (detector)

vs.: versus

WIMP: Weak Interaction Massive Particle

XML: eXtensible Markup Language 

$\Lambda CDM$: Lambda Cold Dark Matter
